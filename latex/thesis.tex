%%%%%%%%%%%%%%%%%%%%%%%%%%%%%%%%%%%%%%%%%%%%%%%%%%%%%%%%%%%%%%%%%%%%%%%%%%%%%%%%
%%%%%%%%%%%%%%%%%%%%%%%%%%%%%%%%%%%%%%%%%%%%%%%%%%%%%%%%%%%%%%%%%%%%%%%%%%%%%%%%
%%                                                                            %%
%% thesistemplate.tex version 4.00 (2023/02/09)                               %%
%% The LaTeX template file to be used with the aaltothesis.sty (version 4.00) %%
%% style file.                                                                %%
%% This package requires pdfx.sty v. 1.5.84 (2017/05/18) or newer.            %%
%%                                                                            %%
%% This is licensed under the terms of the MIT license below.                 %%
%%                                                                            %%
%% Written by Luis R.J. Costa.                                                %%
%% Currently developed at Teacher services, Learning Services of Aalto        %%
%% University by Luis R.J. Costa since May 2019.                              %%
%%                                                                            %%
%% Copyright 2017-2021 aaltothesis.cls by Luis R.J. Costa,                    %%
%% luis.costa@aalto.fi.                                                       %%
%% Copyright 2017-2018 Swedish translations in aaltothesis.cls by Elisabeth   %%
%% Nyberg and Henrik Wallén henrik.wallen@aalto.fi.                           %%
%% Finnish documentation in the template opinnatepohja.tex translated from    %%
%% the English template documentation.                                        %%
%% Copyright 2021 English template thesistemplate.tex by Luis R.J. Costa,     %%
%% Maurice Forget, Henrik Wallén.                                             %%
%% Copyright 2018-2022 Swedish template kandidatarbetsbotten.tex by Henrik    %%
%% Wallen.                                                                    %%
%%                                                                            %%
%% Permission is hereby granted, free of charge, to any person obtaining a    %%
%% copy of this software and associated documentation files (the "Software"), %%
%% to deal in the Software without restriction, including without limitation  %%
%% the rights to use, copy, modify, merge, publish, distribute, sublicense,   %%
%% and/or sell copies of the Software, and to permit persons to whom the      %%
%% Software is furnished to do so, subject to the following conditions:       %%
%% The above copyright notice and this permission notice shall be included in %%
%% all copies or substantial portions of the Software.                        %%
%% THE SOFTWARE IS PROVIDED "AS IS", WITHOUT WARRANTY OF ANY KIND, EXPRESS OR %%
%% IMPLIED, INCLUDING BUT NOT LIMITED TO THE WARRANTIES OF MERCHANTABILITY,   %%
%% FITNESS FOR A PARTICULAR PURPOSE AND NONINFRINGEMENT. IN NO EVENT SHALL    %%
%% THE AUTHORS OR COPYRIGHT HOLDERS BE LIABLE FOR ANY CLAIM, DAMAGES OR OTHER %%
%% LIABILITY, WHETHER IN AN ACTION OF CONTRACT, TORT OR OTHERWISE, ARISING    %%
%% FROM, OUT OF OR IN CONNECTION WITH THE SOFTWARE OR THE USE OR OTHER        %%
%% DEALINGS IN THE SOFTWARE.                                                  %%
%%                                                                            %%
%%                                                                            %%
%%%%%%%%%%%%%%%%%%%%%%%%%%%%%%%%%%%%%%%%%%%%%%%%%%%%%%%%%%%%%%%%%%%%%%%%%%%%%%%%
%%                                                                            %%
%%                                                                            %%
%% An example for writting your thesis using LaTeX                            %%
%% Original version and development work by Luis Costa, changes to the text   %% 
%% in the Finnish template by Perttu Puska.                                   %%
%% Support for Swedish added 15092014                                         %%
%% PDF/A-b support added on 15092017                                          %%
%% PDF/A-2 support added on 24042018                                          %%
%% Layout design and typesettin changed 15072021                              %%
%%                                                                            %%
%% This example consists of the files                                         %%
%%       thesistemplate.tex (version 4.00) (for text in English)              %%
%%       opinnaytepohja.tex (version 4.00) (for text in Finnish)              %%
%%       kandidatarbetsbotten.tex (version 1.10) (for text in Swedish)        %%
%%       aaltothesis.cls                                                      %%
%%       linediagram.pdf (graphics file)                                      %%
%%       curves.pdf      (graphics file)                                      %%
%%       ledspole.jpg    (graphics file)                                      %%
%%                                                                            %%
%%                                                                            %%
%% Typeset in Linux with                                                      %%
%% pdflatex: (recommended method)                                             %%
%%             $ pdflatex thesistemplate                                      %%
%%             $ pdflatex thesistemplate                                      %%
%%                                                                            %%
%%   The result is the file thesistemplate.pdf that is PDF/A compliant, if    %%
%%   you have chosen the proper \documenclass options (see comments below)    %%
%%   and your included graphics files have no problems.                       %%
%%                                                                            %%
%%                                                                            %%
%% Explanatory comments in this example begin with the characters %%, and     %%
%% changes that the user can make with the character %                        %%
%%                                                                            %%
%%%%%%%%%%%%%%%%%%%%%%%%%%%%%%%%%%%%%%%%%%%%%%%%%%%%%%%%%%%%%%%%%%%%%%%%%%%%%%%%
%%%%%%%%%%%%%%%%%%%%%%%%%%%%%%%%%%%%%%%%%%%%%%%%%%%%%%%%%%%%%%%%%%%%%%%%%%%%%%%%
%%
%% WHAT is PDF/A
%%
%% PDF/A is the ISO-standardized version of the pdf. The standard's goal is to
%% ensure that he file is reproducable even after a long time. PDF/A differs
%% from pdf in that it allows only those pdf features that support long-term
%% archiving of a file. For example, PDF/A requires that all used fonts are
%% embedded in the file, whereas a normal pdf can contain only a link to the
%% fonts in the system of the reader of the file. PDF/A also requires, among
%% other things, data on colour definition and the encryption used.
%% Currently three PDF/A standards exist:
%% PDF/A-1: based on PDF 1.4, standard ISO19005-1, published in 2005.
%%          Includes all the requirements essential for long-term archiving.
%% PDF/A-2: based on PDF 1.7, standard ISO19005-2, published in 2011.
%%          In addition to the above, it supports embedding of OpenType fonts,
%%          transparency in the colour definition and digital signatures.
%% PDF/A-3: based on PDF 1.7, standard ISO19005-3, published in 2012.
%%          Differs from the above only in that it allows embedding of files in
%%          any format (e.g., xml, csv, cad, spreadsheet or wordprocessing
%%          formats) into the pdf file.
%% PDF/A-4: based on PDF 2.0, standard ISO19005-4, published in November 2020.
%%
%% PDF/A-1 files are not necessarily PDF/A-2 -compatible and PDF/A-2 are not
%% necessarily PDF/A-1 -compatible.
%% Standards PDF/A-1, PDF/A-2 and PDF/A-3 have two levels:
%% b: (basic) requires that the visual appearance of the document is reliably
%%    reproduceable.
%% a (accessible) in addition to the b-level requirements, specifies how
%%   accessible the pdf file is to assistive software, say, for the physically
%%   impaired.
%% The PDF/A-4 standard does not have additional levels like in the earlier
%% standards.
%% For more details on PDF/A, see, e.g., 
%% https://www.loc.gov/preservation/digital/formats/fdd/fdd000318.shtml or
%% https://www.pdfa.org/resource/iso-19005-pdfa/
%%
%%
%% WHICH PDF/A standard should my thesis conform to?
%%
%% Either to the PDF/A-1b or the PDF/A-2b standard. If all the figures and
%% graphs used in thesis work do not require transparency features, use either
%% PDF/A-1b or PFDF/A-2b. If you have figures with transparency
%% characteristics, use the PDF/A-2b standard. However, drawing applications
%% often use the transparency parameter, setting it to zero, to specify opacity
%% and get the basic 2-D visualisation. As a result, validation of PDF/A-1b
%% will fail. Use PDF/A-2b if PDF/A-1b validation fails.
%% Do not use the PDF/A-3b standard for your thesis.
%% The font to be used are specified in this templatenand they should not be
%% changed. In addition to not adhering to Aalto's visual guidelines, you may
%% have difficulties in producing a PDF/A-compliant PDF.
%%
%%
%% Validate your PDF/A file at https://www.pdf-online.com/osa/validate.aspx
%%
%%
%% WHAT graphics format can I use to produce my PDF/A compliant file?
%%
%% When using pdflatex to compile your work, favour the use of pdf, but you can
%% use the jpg or png format especially for photographs. You will have PDF/A 
%% compliance problems with figures in pdf if the fonts are not embedded in the
%% pdf file.
%% If you choose to use latex to compile your work, the only acceptable file
%% format for your figure is eps. DO NOT use the ps format for your figures.

%% USE one of the following three \documentclass set-ups:
%% * the first when using pdflatex to directly typeset your document in the
%%   chosen pdf/a format for online publishing (centred page layout),
%% * the second for one-sided printing your thesis with the layout (wide left 
%%   margin), or
%% * the third for two-sided printing.
%%
\documentclass[english, 12pt, a4paper, sci, utf8, a-2b, online]{aaltothesis}
%\documentclass[english, 12pt, a4paper, elec, utf8, a-2b, print]{aaltothesis}
%\documentclass[english, 12pt, a4paper, elec, utf8, a-2b, print, twoside]{aaltothesis}

%% Use the following options in the \documentclass macro above:
%% your school: arts, biz, chem, elec, eng, sci
%% the character encoding scheme used by your editor: utf8, latin1
%% thesis language: english, finnish, swedish
%% make an archiveable PDF/A-1b or PDF/A-2b compliant file: a-1b, a-2b
%%                    (with pdflatex, a normal pdf containing metadata is
%%                     produced without the a-*b option)
%% typset for online document or print on paper: online, print
%%        online: typeset in symmetric layout and blue hypertext for online
%%                publishing
%%        print: typeset in a symmetric layout and black hypertext for printing
%%               on paper
%%          two-side printing: twoside (default is one-sided printing)
%%               typeset in a wide margin on the binding side of the page and
%%               black hypertext. Use with print only.
%%

%% Use one of these if you write in Finnish (or use the Finnish template
%% opinnaytepohja.tex)
%\documentclass[finnish, 12pt, a4paper, elec, utf8, a-1b, online]{aaltothesis}
%\documentclass[finnish, 12pt, a4paper, elec, utf8, a-1b, print]{aaltothesis}
%\documentclass[finnish, 12pt, a4paper, elec, utf8, a-1b, print, twoside]{aaltothesis}

%% Use one of these if you write in Swedish (or use the Swedish template
%% kandidatarbetsbotten.tex)
%\documentclass[swedish, 12pt, a4paper, elec, utf8, a-2b, online]{aaltothesis}
%\documentclass[swedish, 12pt, a4paper, elec, utf8, a-2b]{aaltothesis}
%\documentclass[swedish, 12pt, a4paper, elec, dvips, online]{aaltothesis}

%% FOR USERS OF AMS PACKAGES:
%% * newtxmath used in this template loads amsmath, so
%%   you needn't load it. If you want to use options in amsmath, load it here, 
%%   before \setupthesisfonts below to pass the options to amsmath.
%% * If you want to use amsthm, load it here before \setupthesisfonts to avoid
%%   a clash with newtxmath.
%% * If using amsmath with options and you want to use amsthm, load amsthms
%%   after amsmath, as described in the amsthm documentation.
%% * Don't use amsbsym or amsfonts. The symbols [and macros] there are defined in
%%   newtxmath and so clash if used.
%\usepackage[options]{amsmath}
%\usepackage{amsthm}

%% DO NOT MOVE OR REMOVE \setupthesisfonts
\setupthesisfonts

%%
%% Add here the packges you need
%%
\usepackage{graphicx}


%% For tables that span multiple pages; used to split a paraphrasing example in
%% the appendix. If you don't need it, remove it.
\usepackage{longtable}

%% A package for generating Creative Commons copyright terms. If you don't use
%% the CC copyright terms, remove it, since otherwise undesired information may
%% be added to this document's metadata.
\usepackage[type={CC}, modifier={by-nc-sa}, version={4.0}]{doclicense}
%% Find below three examples for typesetting the CC license notice.


%% Edit to conform to your degree programme
%% Capitalise the words in the name of the degree programme: it's a name
\degreeprogram{Mathematics and Operations Research}
%%

%% Your major
%%
\major{Applied Mathematics}
%%

%% Choose one of the three below
%%
%\univdegree{BSc}
\univdegree{MSc}
%\univdegree{Lic}
%%

%% Your name (self explanatory...)
%%
\thesisauthor{Joonas Laaksonen}
%%

%% Your thesis title and possible subtitle comes here and possibly, again,
%% together with the Finnish or Swedish abstract. Do not hyphenate the title
%% (and subtitle), and avoid writing too long a title. Should LaTeX typeset a
%% long title (and/or subtitle) unsatisfactorily, you might have to force a
%% linebreak using the \\ control characters. In this case...
%% * Remember, the title should not be hyphenated!
%% * A possible 'and' in the title should not be the last word in the line; it
%%   begins the next line.
%% * Specify the title (and/or subtitle) again without the linebreak characters
%%   in the optional argument in box brackets. This is done because the title
%%   is part of the metadata in the pdf/a file, and the metadata cannot contain
%%   linebreaks.
%%
\thesistitle{Pointwise convergence of high-order finite element
solutions to the Poisson problem with a concentrated load}
%\thesistitle[Title of the thesis]{Title of\\ the thesis}
%%
%% Either remove or leave \thesissubtitle{} empty if you don't use it
%%
%\thesissubtitle{A possible subtitle}
%\thesissubtitle[Subtitle of the thesis]{Subtitle of\\ the thesis}
%\thesissubtitle{}

%%
\place{Espoo}
%%

%% The date for the bachelor's thesis is the day it is presented
%%
\date{9 February 2023}
%%

%% Thesis supervisor
%% Note the "\" character in the title after the period and before the space
%% and the following character string.
%% This is because the period is not the end of a sentence after which a
%% slightly longer space follows, but what is desired is a regular interword
%% space.
%%
%\supervisor{Prof.\ Pirjo Professori}
\supervisor{D.Sc.\ (Tech.) Harri Hakula}
%%

%% Advisor(s)---two at the most---of the thesis. Check with your supervisor how
%% many official advisors you can have.
%%
%\advisor{Dr Alan Advisor}
%\advisor{Ms Elsa Expert (MSc)}
%%

%% If you do your thesis work in a company of other institute, give the name of
%% the company or instution here. Otherwise, leave the macro empty, comment it
%% out, or remove it. This will remove this field from the abstract page.
%%
%\collaborativepartner{Company or institute name}
%%

%% Aaltologo: syntax:
%% \uselogo{?|!|''}
%% The logo language is set to be the same as the thesis language.
%%
%\uselogo{?}
%\uselogo{!}
\uselogo{''}
%%

%%%%%%%%%%%%%%%%%%               COPYRIGHT TEXT               %%%%%%%%%%%%%%%%%%
%%%%%%%%%%%%%%%%%%%%%%%%%%%%%%%%%%%%%%%%%%%%%%%%%%%%%%%%%%%%%%%%%%%%%%%%%%%%%%%%

%% Copyright of a work is with the creator/author of the work regardless of
%% whether the copyright mark is explicitly in the work or not. You may, if you
%% wish---we encourage you to do so---publish your work under a Creative
%% Commons license (see creativecommons.org), in which case the license text
%% must be visible in the work. Write here the copyright text you want using the
%% macro \copyrighttext, which writes the text into the metadata of the pdf file
%% as well.
%%
%% Syntax:
%% \copyrigthtext{metadata text}{text visible on the page}
%%
%% CHOOSE ONE OF THE COPYRIGHT NOTICE STYLES BELOW.
%% IF USING THE CC TERMS, CHOOSE THE LICENSE YOU WANT TO USE.
%% The different CC licenses are listed at 
%% https://creativecommons.org/about/cclicenses/.
%% If you use the icons from the dolicense.sty package, add the package above
%% (\usepackage{dolicense}).
%% IMPORTANT NOTE!! Manually write the CC text in the \copyrighttext metadata
%% text field.
%%
%% NOTE: In the macros below, the text written in the metadata must have a
%% \noexpand macro before the \copyright special character. When not in pdf/a
%% mode (i.e. a-1b or a-2b are not specified in \documentclass), two \noexpands
%% are required in the metadata text to correctly render the copyright mark in
%% the pdf metadata. In pdf/a mode one \noexpand suffices.
%%
%% EXAMPLE OF PLAIN COPYRIGHT TEXT
%% The macros \copyright and \year below must be separated by the \ character 
%% (space chacter) from the text that follows. The macros in the argument of the
%% \copyrighttext macro automatically insert the year and the author's name.
%% (Note! \ThesisAuthor is an internal macro of the aaltothesis.cls class file).
%%
%\copyrighttext{Copyright \noexpand\textcopyright\ \number\year\ \ThesisAuthor}
%{Copyright \textcopyright{} \number\year{} \ThesisAuthor}
%%
%% Of course, the same text could have simply been written as
%% \copyrighttext{Copyright \noexpand\copyright\ 2018 Eddie Engineer}
%% {Copyright \copyright{} 2022 Eddie Engineer}
%%
%% EXAMPLES OF CC LICENSE: different ways to display the same license
%% 1. A simple Creative Commons license text with a link to the copyright notice:
%\copyrighttext{\noexpand\textcopyright\ \number\year. This work is 
%	licensed under a CC BY-NC-SA 4.0 license.}{\textcopyright{} 
%	\number\year. This work is licensed under a 
%	\href{https://creativecommons.org/licenses/by-nc-nd/4.0/}{CC BY-NC-SA 4.0} 
%	license.}
%
%% To get the URL of the license of your choice, go to 
%% https://creativecommons.org/about/cclicenses/, click on the chosen license
%% you want to use, and copy-and-paste the URL in the macro \href above.
%%
%% 2. A short Creative Commons license text containing the respective CC icons
%% (requires the package dolicense.sty to be added in the preamble as done
%% above) and a link to the corresponding Creative Commons license webpage (see
%% the dolicense package documentation for other license icons):
%\copyrighttext{\noexpand\textcopyright\ \number\year. This work is licensed
%	under a CC BY-NC-SA 4.0 license.}{
%	\parbox{95mm}{\noindent\textcopyright\ \number\year. \doclicenseText} 
%	\hspace{1em}\parbox{35mm}{\doclicenseImage}
%}
%%
%% 3. An expanded Creative Commons license text containing the respective CC
%% icons text and as generated by the dolicense.sty package (the license is set
%% via package options in \usepackage[options]{dolicense} above; see the
%% dolicense package documentation for other license texts and icons):
\copyrighttext{\noexpand\textcopyright\ \number\year. This work is 
	licensed under a Creative Commons "Attribution-NonCommercial-ShareAlike 4.0 
	International" (BY-NC-SA 4.0) license.}{\noindent\textcopyright\ \number
	\year \ \doclicenseThis}
%%%%%%%%%%%%%%%%%%%%%%%%%%%%%%%%%%%%%%%%%%%%%%%%%%%%%%%%%%%%%%%%%%%%%%%%%%%%%%%%


%% The English abstract:
%% All the details (name, title, etc.) on the abstract page appear as specified
%% above.
%% Thesis keywords:
%% Note! The keywords are separated using the \spc macro
%%
\keywords{For keywords choose\spc concepts that are\spc central to your\spc 
thesis}
%%

%% The abstract text. This text is included in the metadata of the pdf file as
%% well as the abstract page.
%%
\thesisabstract{%
The abstract is a short description of the essential contents of the thesis:
what was studied and how and what were the main findings. For a Finnish thesis,
the abstract should be written in both Finnish and English; for a Swedish
thesis, in Swedish and English. The abstracts for English theses written by
Finnish or Swedish speakers should be written in English and either in Finnish
or in Swedish, depending on the student’s language of basic education. Students
educated in languages other than Finnish or Swedish write the abstract only in
English. Students may include a second or third abstract in their native
language, if they wish. 
The abstract text of this thesis is written on the readable abstract page as
well as into the pdf file's metadata via the thesisabstract macro (see the 
comment in the TeX file). Write here the text that goes into the metadata. The 
metadata cannot contain special characters, linebreak or paragraph break 
characters, so these must not be used here. If your abstract does not contain 
special characters and it does not require paragraphs, you may take advantage of
the abstracttext macro (see the comment in the TeX file below). Otherwise, the 
metadata abstract text must be identical to the text on the abstract page.
}

%% You can prevent LaTeX from writing into the xmpdata file (it contains all the 
%% metadata to be written into the pdf file) by setting the writexmpdata switch
%% to 'false'. This allows you to write the metadata in the correct format
%% directly into the file thesistemplate.xmpdata.
%\setboolean{writexmpdatafile}{false}


%% All that is printed on paper starts here
%%
\begin{document}

%% Create the coverpage
%%
\makecoverpage

%% Typeset the copyright text.
%% If you wish, you may leave out the copyright text from the human-readable
%% page of the pdf file. This may seem like a attractive idea for the printed
%% document especially if "Copyright (c) yyyy Eddie Engineer" is the only text
%% on the page. However, the recommendation is to print this copyright text.
%%
\makecopyrightpage

\clearpage
%% Note that when writing your thesis in English, place the English abstract
%% first followed by the possible Finnish or Swedish abstract.

%% Abstract text
%% All the details (name, title, etc.) on the abstract page appear as specified
%% above.
%%
\begin{abstractpage}[english]
  The abstract is a short description of the essential contents of the thesis:
  what was studied and how and what were the main findings.

  For a Finnish thesis, the abstract should be written in both Finnish and
  English; for a Swedish thesis, in Swedish and English. The abstracts for
  English theses written by Finnish or Swedish speakers should be written in
  English and either in Finnish or in Swedish, depending on the student’s
  language of basic education. Students educated in languages other than Finnish
  or Swedish write the abstract only in English. Students may include a second
  or third abstract in their native language, if they wish.

  The abstract text of this thesis is written on the readable abstract page as
  well as into the pdf file's metadata via the \verb+\thesisabstract+ macro
  (see comment in this \TeX{} file above). Write here the text that goes onto
  the readable abstract page. You can have special characters, linebreaks, and
  paragraphs here. Otherwise, this abstract text must be identical to the
  metadata abstract text.
  
  If your abstract does not contain special characters and it does not require
  paragraphs, you may take advantage of the \verb+\abstracttext+ macro (see the
  comment in this \TeX{} file below).
\end{abstractpage}

%% The text in the \thesisabstract macro is stored in the macro \abstractext, so
%% you can use the text metadata abstract directly as follows:
%%
%\begin{abstractpage}[english]
%	\abstracttext{}
%\end{abstractpage}

%% Force a new page so that the possible Finnish or Swedish abstract does not
%% begin on the same page
%%
\newpage
%%
%% Abstract in Finnish.  Delete if you don't need it. 
%%
%% Respecify those fields that differ from the earlier specification or simply
%% respecify all fields.
\thesistitle{Opinnäyteen otsikko}
%\thesissubtitle{Opinnäytteen mahdollinen alaotsikko}
\supervisor{TkT Harri Hakula}
%\advisor{TkT Alan Advisor}
%\advisor{DI Elsa Expert}
\degreeprogram{Matematiikka ja operaatioanalyysi}
\major{Sovellettu matematiikka}
%\collaborativepartner{Yhtiön tai laitoksen nimi}
\date{9.2.2023}
%% The keywords need not be separated by \spc now.
\keywords{Vastus, resistanssi, lämpötila}
%% Abstract text
\begin{abstractpage}[finnish]
Tiivistelmä on lyhyt kuvaus työn keskeisestä sisällöstä: mitä tutkittiin ja 
miten sekä mitkä olivat tärkeimmät tulokset. Suomenkielisen opinnäytteen 
tiivistelmä kirjoitetaan suomeksi ja englanniksi ja ruotsinkielisen vastaavasti 
ruotsiksi ja englanniksi. Suomen- tai ruotsinkielisten opiskelijoiden, joiden 
opinnäytteen kieli on englanti, tulee kirjoittaa tiivistelmänsä englanniksi ja 
koulusivistyskielellään. Muiden kuin koulusivistyskieleltään suomen- tai 
ruotsinkielisten tulee kirjoittaa tiivistelmänsä vain englanniksi. Opiskelija 
voi halutessaan lisätä opinnäytteeseensä toisen tai kolmannen tiivistelmän 
omalla äidinkielellään.
Tämän opinnäytteen tiivistelmäteksti kirjoitetaan opinnäytteen luettavan osan
lomakkeen lisäksi myös pdf-tiedoston metadataan. Kirjoita tähän metadataan
kirjoitettavaa teksti. Metadatatekstissa ei saa olla erikoismerkkejä,
rivinvaiho- tai kappaleenjakomerkkiä, joten näitä merkkeja ei saa käyttää tässä.
Jos tiivistelmäsi ei sisällä erikoimerkkejä eikä kaipaa kappaleenjakoa, voit
hyödynttää makroa abstracttext luodessasi lomakkeen tiivistelmää (katso
kommentti tässä TeX-tiedostossa alla). Metadatatiivistelmatekstin on muuten 
oltava sama kuin lomakkeessa oleva teksti.

\end{abstractpage}

\dothesispagenumbering{}

\newpage

%% Table of contents. 
%%
\thesistableofcontents

%% \clearpage is similar to \newpage, but it also flushes the floats (figures
%% and tables).
%%
\cleardoublepage

\section{Introduction}
\label{sec:intro}

%% Leave page number of the first page empty
%% 
\thispagestyle{empty}
A partial differential equation (PDE) is an equation that consists of an
unknown function of two or more variables and its partial derivatives of
arbitrary order \cite{evans2010}.
Such equations describe how a function, possibly corresponding to a physical
quantity of interest, behaves over its domain of definition
which in practical applications typically corresponds to a geometric shape or
time or both. For example, many fundamental laws of physics can be elegantly
expressed as partial differential equations, e.g.\ Maxwell's equations
in electromagnetism.
The set of all possible partial differential equations is incredibly vast and
complex, which makes it unwieldy, and most probably impossible,
to come up with a general PDE theory that could be productively used for any kind
of problem. Instead, the study of partial differential equations
focuses on important families and instances of PDEs arising from
different fields of science. For example, Poisson's 
equation belongs to the family of elliptic partial differential equations.

A solution of a partial differential equation is said to be classical if it
can be differentiated at least as many times as the formulation of the PDE
requires and the PDE holds pointwise everywhere in its domain.
Now one may wonder how and why a solution should be characterized in
any other way, but it turns out that rather few partial differential equations
have classical solutions. Moreover, imposing boundary conditions,
that the possible classical solution must satisfy on the boundary of its domain,
complicates the question of existence even further.
For some phenomena the solution may even be expected to be non-differentiable
at some points, so it makes sense to look for solutions in some other sense as well
than just classical sense. Thus, the initial partial differential equation is
usually reformulated in a more generalized form, which essentially expands
the space of functions where the solution is searched and enables a set of useful
mathematical techniques to be utilized. A solution of the
generalized problem is usually called a generalized solution or a weak solution,
and its existence can be guaranteed for a large set of problems.
Whether a weak solution is also a classical solution can then be
assessed separately, and such results fall under the subdiscipline of
regularity theory of PDEs.

In practical applications numerical methods are used to approximate the
solutions of partial differential equations. One such method that has been
extremely successful is the finite element method (FEM). Oden \cite{oden1991}
provides a review of its history. Without too stringent a viewpoint, some attributes 
of the finite element method can be traced back a couple of centuries, but serious
interest in the method started to accumulate during the mid-1950s and 1960s
especially in the engineering community. During this time the finite element method 
also gained its name. The mathematical foundations were established somewhat
later during the 1970s, and the convergence of the method was assessed for several
classes of problems.

Oden also discusses some of the factors leading to the success of the finite
element method. The method is based on the generalized, weak formulation of a
partial differential equation which will be discussed in more
detail in later sections, but for now it suffices to say that it is a crucial
factor why the finite element method merits its success. Being based on the weak
formulation, the finite element method is essentially geometry-agnostic, which
means that it can be used to solve partial differential equations over almost any
kind of shape. Combined with the rich modern theory of partial differential
equations, the finite element method has solid mathematical foundations which offer
optimal estimates of the convergence of the approximations. 
From a computational standpoint, the implementation of the method in 
computer code lends itself extremely well to parallelization.

Decision-making based on computed information requires that the computed
information is reliable. Szabó and Babu{\v s}ka \cite{szabobabuska2011}
discuss this systematically in the context of finite element analysis, i.e.\
the process of using the finite element method to solve a problem. In finite
element analysis the typical workflow is to create a mathematical model, i.e.\ a set 
of partial differential equations and constraints,
that represents the physical system of interest,
find an approximate solution to the model by using the finite element method,
and finally extract the desired computed information from the approximate solution.
There are two critical factors contributing to the reliability of such computed 
information: the suitability of the mathematical model as a representation of
idealized reality and the accuracy of the approximated solution with respect
to the exact solution of the mathematical model.
Szabó and Babu{\v s}ka call the processes of assessing these qualities validation 
and verification, respectively.
The validation process may consist of, for example, comparing the results of
real-life experiments to predictions obtained from the mathematical model.
The verification process leans on the well-understood approximation
properties of the finite element method for many practical problems.

Sometimes a physical phenomenon may be modeled sufficiently well by a 
mathematical model for which the approximation results from the standard
theory of the finite element method do not apply, at least directly.
If possible, one option to still be able to perform the verification process would 
be to change the model so that the error estimates readily apply.
However, there could be some tradeoffs involved in the choice between the models,
which could still make the initial model more favorable,
e.g.\ the initial model is a crude simplification but requires much less
effort to solve. For example, Babu{\v s}ka et al.\ \cite{babuskasoanesuri2017}
study the effects of replacing holes having extremely small radii with singular
points, i.e.\ holes with radii zero, which simplifies the approximation process
but essentially renders the model as incorrect but still useful.

In this thesis we study a problem similar in vein to the previous paragraph
in the sense that convergence of the finite element method is not obvious.
The problem is Poisson's equation with Dirac delta source term, and it can
be expressed as finding a function $u$ such that
\begin{equation*}
    -\Delta u = \delta_{x_0},
\end{equation*}
where $\Delta u$ is the Laplacian of $u$,
i.e.\ the sum of second partial derivatives of $u$ with respect to each independent
variable, and $\delta_{x_0}$ is the Dirac delta distribution which can be thought
of as a measure that evaluates a given function at the point $x_0$ when integrated
against. The presence of the Dirac delta means that the equality above is to be
understood in some specific sense, namely in the weak sense,
which will be made precise in later sections. Poisson's equation can be used to
model, for example, the electrostatic potential caused by an electric charge, and
with the Dirac delta source term the equation can be thought to model an idealized
point charge, i.e.\ a charge with no volume.

We study the above problem in convex polygonal two dimensional domains
and with different boundary conditions, namely
Dirichlet, Neumann and mixed Dirichlet-Neumann boundary conditions.
The first objective of this thesis is to study the unique solvability of these
problems. Casas \cite{casas1985} proves this for the Dirichlet problem with
homogeneous boundary values. We extend this result to the other boundary 
conditions as well with non-homogeneous values.
The second objective is to study the convergence
of the finite element method in specific norms when applied to these problems.
We shall consider the h- and p-versions of the finite element method which will
be presented later. Casas \cite{casas1985} and Scott \cite{scott1973}
show convergence for the h-version in $L^2$ integral norm.
Schatz and Wahlbin \cite{schatzwahlbin1977} also prove estimates for pointwise
convergence for the h-version. We extend the result of $L^2$ convergence
by Casas to the p-version. We also assess the convergence numerically
for a Neumann problem with emphasis on pointwise convergence which sometimes
turns out to be exponential. Finally, the Dirac delta distribution is replaced
by its derivative in which case the convergence of the finite element method is     
studied numerically. The Dirac delta distribution and its derivative are jointly
referred to as concentrated loads when used as the source terms in Poisson's 
equation. More recent results have been obtained by Millar et al.\
\cite{millarmuga2021} who obtain convergence by approximating the
Dirac delta function and by Araya et al.\ \cite{arayabehrens2006} who deduce
a posteriori error estimates.

The remainder of this thesis is structured as follows.

\clearpage
%% Bibliography/ list of references
%%
\thesisbibliography
\bibliographystyle{ieeetr}
\bibliography{refs}

\end{document}
